\chapter{Related Work}

With the advent of mobile technology, museums are exploring new ways to increase visitor engagement in the learning process. Mobile applications and interactive guides are being used to conduct personalized tours, receive information in augmented reality, and more.

\section*{Mobile Applications in Museums}

Several projects have demonstrated the potential of mobile applications to improve the museum experience. For example, the development of mobile applications such as the ``Louvre Visit, Tours \& Guide`` \cite{louvre_app} and the ``British Museum Audio`` \cite{british_museum_app} has shown that visitors can benefit from access to a digital guidebook that not only provides additional information about the exhibits, but also shares practical information, navigation, and flexible self-guided tours.

However, to access information, these applications mainly use manual user input, such as scanning QR codes or entering exhibit numbers. Despite its effectiveness, this approach lacks the speed and seamlessness that can be achieved with more advanced technologies such as image recognition and real-time classification.

\section*{Image Classification in Cultural Heritage Institutions}

The integration of image recognition technology into museum applications represents a significant step forward in creating a more interactive and responsive user experience. For example, a project called ``CHESS`` (Cultural Heritage Experiences through Socio-personal interactions and Storytelling) \cite{chess_project}, co-funded by the European Commission, has explored the use of image recognition to identify works of art and provide augmented reality experience. Such systems allow visitors to point their devices at the exhibits to instantly receive detailed information, thereby improving the learning process by minimizing the effort required to interact with the content.

\section*{Summary}

In summary, the integration of mobile applications and image recognition models into the museum experience has demonstrated great potential in increasing visitor engagement and learning. While traditional mobile guides provide valuable information, the addition of real-time image classification represents a significant step forward in providing seamless interaction. As museums continue to introduce these technologies, visitors' experiences are likely to become more personalized, dynamic, and informative.