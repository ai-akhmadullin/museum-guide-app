%%% Please fill in basic information on your thesis, which will be automatically
%%% inserted at the right places. You need to replace \xxx{...} by real data.

% Type of your thesis:
%	"bc" for Bachelor's
%	"mgr" for Master's
%	"phd" for PhD
%	"rig" for rigorosum
\def\ThesisType{bc}

% Language of your study programme:
%	"cs" for Czech
%	"en" for English
\def\StudyLanguage{cs}

% Thesis title in English (exactly as in the official assignment)
% (Note: \xxx is a "ToDo label" which makes the unfilled visible. Remove it.)
\def\ThesisTitle{{Mobile app for real-time exhibit recognition in a museum}}

% Author of the thesis (you)
\def\ThesisAuthor{{Airat Akhmadullin}}

% Year when the thesis is submitted
\def\YearSubmitted{{2025}}

% Name of the department or institute, where the work was officially assigned
% (according to the Organizational Structure of MFF UK in English,
% see https://www.mff.cuni.cz/en/faculty/organizational-structure,
% or a full name of a department outside MFF)
\def\Department{{Department of Software and Computer Science Education}}

% Is it a department (katedra), or an institute (ústav)?
\def\DeptType{{Department}}

% Thesis supervisor: name, surname and titles
\def\Supervisor{{doc. RNDr. Elena Šikudová, Ph.D.}}

% Supervisor's department (again according to Organizational structure of MFF)
\def\SupervisorsDepartment{{Department of Software and Computer Science Education}}

% Study programme (does not apply to rigorosum theses)
\def\StudyProgramme{{Computer Science}}

% An optional dedication: you can thank whomever you wish (your supervisor,
% consultant, who provided you with tea and pizza, etc.)
\def\Dedication{%
{I would like to thank my supervisor, Elena Šikudová, for her guidance and support, 
and my family for their constant encouragement throughout this journey.}
}

% Abstract (recommended length around 80-200 words; this is not a copy of your thesis assignment!)
\def\Abstract{%
{This thesis presents the development of a mobile application aimed at enhancing the museum 
experience by providing real-time image classification of exhibits. The application utilizes 
a fine-tuned MobileNet neural network, trained on frames extracted from video recordings 
of each exhibit. The model is converted to TensorFlow Lite format for efficient on-device 
inference. Users can point their mobile device's camera at an exhibit, and the application 
automatically captures images at regular intervals for classification. Upon recognizing 
an exhibit, the app retrieves and displays additional information, offering an interactive 
and informative guide for museum visitors.}
}

% 3 to 5 keywords (recommended) separated by \sep
% Keywords are useful for indexing and searching for the theses by topic.
\def\ThesisKeywords{%
{Computer vision\sep{}Deep learning\sep{}Mobile app\sep{}Image classification\sep{}Android}
}

% If any of your metadata strings contains TeX macros, you need to provide
% a plain-text version for use in XMP metadata embedded in the output PDF file.
% If you are not sure, check the generated thesis.xmpdata file.
\def\ThesisAuthorXMP{\ThesisAuthor}
\def\ThesisTitleXMP{\ThesisTitle}
\def\ThesisKeywordsXMP{\ThesisKeywords}
\def\AbstractXMP{\Abstract}

% If your abstracts are long and do not fit in the infopage, you can make the
% fonts a bit smaller by this setting. (Also, you should try to compress your abstract more.)
\def\InfoPageFont{}
%\def\InfoPageFont{\small}  % uncomment to decrease font size

% If you are studing in a Czech programme, you also need to provide metadata in Czech:
% (in English programmes, this is not used anywhere)

\def\ThesisTitleCS{{Mobilní aplikace pro rozpoznávání exponátů v reálném čase v muzeu}}
\def\DepartmentCS{{Katedra softwaru a výuky informatiky}}
\def\DeptTypeCS{{Katedra}}
\def\SupervisorsDepartmentCS{{Katedra softwaru a výuky informatiky}}
\def\StudyProgrammeCS{{Informatika}}

\def\ThesisKeywordsCS{%
{Počítačové vidění\sep Hluboké učení\sep Mobilní aplikace\sep Klasifikace obrazu\sep Android}
}

\def\AbstractCS{%
{Tato práce představuje vývoj mobilní aplikace zaměřené na vylepšení zážitku z návštěvy muzea 
poskytováním klasifikace obrazů exponátů v reálném čase. Aplikace využívá jemně vyladěnou 
neuronovou síť MobileNet, vycvičenou na snímcích extrahovaných z videozáznamů jednotlivých 
exponátů. Model je převeden do formátu TensorFlow Lite pro efektivní inference přímo na zařízení. 
Uživatelé mohou namířit kameru svého mobilního zařízení na exponát a aplikace automaticky 
v pravidelných intervalech zachytí snímky pro klasifikaci. Po rozpoznání exponátu aplikace načte 
a zobrazí další informace, čímž nabízí interaktivního a informativního průvodce pro návštěvníky 
muzea.}
}
