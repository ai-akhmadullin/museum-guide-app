\chapter{User Manual}

This chapter provides instructions on how to use the mobile application effectively.

\section{Installation}

To install the application:

\begin{enumerate}
\item Transfer the provided APK file (located in the \texttt{MuseumGuideApp} folder under the name \texttt{museum-guide.apk}) to your Android smartphone.
\item Navigate to the transferred file using a file manager app.
\item Tap on the APK file to start installation. You might need to enable \textbf{Install from unknown sources} if prompted.
\item After installation, open the application from your home screen or app drawer.
\end{enumerate}

\section{Navigating the Application}

Upon launching the application, you will see the \textbf{Home Screen}, displaying general information about the museum. You can scroll down to find links to the museum's social media and contact information.

Navigate using the bottom menu with the following options:

\begin{itemize}
\item \textbf{Home}: Returns you to the Home Screen.
\item \textbf{Camera}: Opens the camera for real-time exhibit recognition.
\item \textbf{Gallery}: Displays all museum exhibits in a scrollable grid.
\end{itemize}

\section{Viewing Exhibits in Gallery}

In the \textbf{Gallery Screen}, you can scroll through and tap on any exhibit to view detailed information such as the title, author, and historical period.

\section{Real-Time Exhibit Recognition}

To use the real-time recognition feature:

\begin{enumerate}
\item Tap on the \textbf{Camera} option in the bottom menu.
\item If prompted, grant camera permissions to the application.
\item Aim your smartphone's camera at an exhibit within the museum. Recognition results will be displayed in real-time, overlaid on the camera feed.
\item Once the model recognizes an exhibit, tap \textbf{See Details} to view more information about the recognized exhibit. If the recognition is incorrect, you can check the alternatives provided below the main prediction. These alternatives are ranked by confidence level and can help you find the correct exhibit.
\end{enumerate}

\section{Debug Mode}

The application includes a Debug Mode for testing purposes:

\begin{enumerate}
\item In the \textbf{Camera Screen}, switch from \textbf{Regural} to \textbf{Debug} mode by tapping the icon in the bottom right corner.
\item Debug Mode allows you to adjust the classification confidence threshold and view detailed classification results, including confidence levels and processing time.
\end{enumerate}

To exit the Debug Mode, tap the icon again.